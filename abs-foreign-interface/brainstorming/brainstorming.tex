\documentclass[a4paper,11pt,final]{article}
%\usepackage{color}
\usepackage[svgnames]{xcolor}
\usepackage{verbatim} 
\usepackage{listing}   % for code listing
\usepackage{listings}[2002/04/01]  %% the new version
\usepackage{colortbl}
\usepackage[final]{graphicx}
\usepackage{subfig}
\usepackage{ifthen}
\usepackage{amsmath}
\usepackage{amssymb}
\usepackage{xspace}
\usepackage{stmaryrd}                    % Exciting symbols
\usepackage{booktabs}      % for fancy tables 
\usepackage{url}

\usepackage{pgf,pgfarrows}                       % PDF drawing package
\usepackage{tikz}
\usetikzlibrary{matrix,arrows}
\usetikzlibrary{shapes}
\usetikzlibrary{patterns}
\usetikzlibrary{positioning}
\usetikzlibrary{calc,decorations.pathreplacing,matrix,shadows,fit} % Compiler figure

%\usetikzlibrary{snakes}
\pgfdeclarelayer{background}
\pgfsetlayers{background,main}

\newtheorem{notation}{Notation}

\lstset{mathescape}
\lstset{basicstyle=\footnotesize\ttfamily}
\lstset{numberstyle=\tiny}%
\lstset{language=}
\lstset{emphstyle=\color{red},emphstyle={[2]\color{blue}\underbar}}%% global

%% ABS
\lstdefinelanguage{ABS}{keywords=
{dyndelta,new,data,type,def,case,of,cog,class,interface,extends,implements,if,else,await,get,Fut,return,skip,while,module, import, export, from, suspend, delta,adds,modifies,removes,original,productline,features,core,corefeatures,optionalfeatures,after,when,product,hasAttribute,hasMethod,root,extension,group,allof,oneof,require,exclude,original,ifin,ifout,opt}, sensitive=true, comment=[l]{//},
  morecomment=[s]{/*}{*/},
  morestring=[b]"}

%\def\codesize{\fontsize{9}{10}}

\lstdefinestyle{absstyle}{
language=ABS,columns=fullflexible,
 		   mathescape=true,%
 		   showstringspaces=false,%
keywordstyle=\bf\sffamily,
commentstyle=\sl\sffamily,%
basicstyle=\footnotesize\tttfamily,
inputencoding=latin1, % i would prefer utf8
extendedchars,xleftmargin=2em
}

\lstnewenvironment{abs}[1][]{%
 \lstset{style=absstyle,#1}%
   \csname lst@SetFirstLabel\endcsname}
  {\csname lst@SaveFirstLabel\endcsname}

% \newcommand{\Abs}[1]{%
%   \lstinline[language=ABS,mathescape=true,%
%            keywordstyle=\bf\sffamily,%
%            basicstyle=\normalsize\sffamily,inputencoding=latin1, % i would prefer utf8
%            extendedchars]!#1!}
%   {\csname lst@SaveFirstLabel\endcsname}

\newcommand{\Abs}[1]{\lstinline[language=ABS,columns=fullflexible,mathescape=true, keywordstyle=\bf\sffamily,basicstyle=\normalsize\ttfamily]!#1!}

\newenvironment{absinline}[1][]{%
\lstset{language=ABS,
		 showstringspaces=false,
		 columns=fullflexible,mathescape=true,%
         keywordstyle=\bf\sffamily,commentstyle=\sl\sffamily,%
         basicstyle=\normalsize\ttfamily,inputencoding=latin1, % i would prefer utf8
         extendedchars,xleftmargin=2em,#1}}
{}

\newcommand{\absshow}[1]{%
\absinline{\lstinputlisting{code/#1}}
}

\newenvironment{abscolumn}{\\\mbox{}\hspace{12pt}\rule{0.95\linewidth}{0.5pt}%\noindent\mbox{}\noindent\hrulefill
\\}{\\\mbox{}\hspace{12pt}\rule{0.95\linewidth}{0.5pt}
% \\\noindent\mbox{}\hrulefill\mbox{}
}

\lstnewenvironment{absexample}{
~\\
\noindent 
\small{\textcolor{Green!80!black}{\bfseries\sffamily{Example:}}}\vspace{-4pt}
\lstset{style=absstyle,
basicstyle=\ttfamily,
framerule=1pt,
backgroundcolor=\color{Green!5},
rulecolor=\color{Green!80!black},
frame=tblr,
xleftmargin=4pt,
}}
{}

\lstnewenvironment{absexamplen}[1][]{
\lstset{style=absstyle,
basicstyle=\ttfamily,
framerule=1pt,
backgroundcolor=\color{Green!5},
rulecolor=\color{Green!80!black},
frame=tblr,
captionpos=b,
xleftmargin=4pt,#1
}}
{}

\lstset{mathescape}
\lstset{basicstyle=\footnotesize\ttfamily}
\lstset{numberstyle=\tiny}%
\lstset{language=}
\lstset{emphstyle=\color{red},emphstyle={[2]\color{blue}\underbar}}%% global

\lstdefinestyle{javastyle}{
language=Java,columns=fullflexible,
 		   mathescape=true,%
 		   showstringspaces=false,%
keywordstyle=\bf\sffamily,
commentstyle=\sl\sffamily,%
basicstyle=\footnotesize\tttfamily,
inputencoding=latin1, % i would prefer utf8
extendedchars,xleftmargin=2em
}

\lstnewenvironment{javaexample}[1][]{
\lstset{style=javastyle,
basicstyle=\ttfamily,
framerule=1pt,
backgroundcolor=\color{Green!5},
rulecolor=\color{Green!80!black},
frame=tblr,
captionpos=b,
xleftmargin=4pt,#1
}}
{}

\newcommand{\mynote}[1]{
  \ensuremath{\left\langle\right.\!\!\left\langle\right.\!}#1%
  \ensuremath{\!\left.\right\rangle\!\!\left.\right\rangle}}
\newcommand{\PW}[1]{\textcolor{blue}{\mynote{PW: #1}}}
\usepackage{amsmath}
\usepackage{amsthm}

\newcommand{\deliverableTitle}{The ABS Foreign Language Interface (ABSFLI)} 

\title{\deliverableTitle}

\author{Jan Sch\"{a}fer \and Peter Y. H. Wong}

\begin{document}

\maketitle 

\section{Introduction}
This document contains first ideas of how to
connect ABS to foreign languages (FLs).
ABS should be able to interact with FLs like
Java to be able to write critical components of a system
in ABS. 

As FLs we mainly consider Java, but further
options are: Maude, Scala, and Erlang

\section{Main questions}
There are essentially two questions to answer:
\begin{enumerate}
  \item How to use FLs from ABS
  \item How to use ABS from Fls
\end{enumerate}

\PW{We should think in terms of functions/datatypes, imperatives and
concurrency?}

\subsection{Concurrency}
\begin{itemize}
  \item \PW{Java object should only invoke ABS method synchronously?}
  \item \PW{Does it make sense to asynchronously invoke Java object? Or both
  ways must be synchronous?}
\end{itemize}

\section{Possible Solutions}

\subsection{Deep Integration (the Scala way)}
One would do a deep integration of Java and treat Java packages as Modules.
For example:

\begin{absexamplen}
import * from java.lang;

{
  Double d = new Double();
}
\end{absexamplen}   

\noindent Advantages: Easy use for ABS to access the Java libraries \\

\noindent Disadvantages
\begin{itemize}
  \item Tightly coupled to Java
  \item Not possible to extend this approach to other languages, e.g. Erlang/C
  ...
  \item Difficult to implement (type-checking)
\end{itemize}
  

\subsection{Loose Integration (the JNI way)}
Use an approach that is similar to JNI. i.e., define interfaces/methods/classes
as \emph{foreign}. Provide an implementation for these interfaces in the
target language 

\noindent Advantages: 
\begin{itemize}
  \item loose coupling
  \item independent of actual language 
\end{itemize}


\noindent Foreign ABS Classes/Functions -- Idea: classes and functions can be
declared to be Foreign, either by using an annotations, or by using some keyword.

\begin{absexamplen}
module Test;

def Int random() = foreign;

interface Output {
  Unit println();
}

[Foreign]
class OuputImpl implements Output { }
\end{absexamplen}

\section{How to link Java to ABS interfaces/classes/functions}

\subsection{Use Conventions}
For example, by having a special naming theme, e.g. a Java class with the same
name as the corresponding ABS class. Example:
     
\noindent ABS:
\begin{absexamplen}
module Foo;
interface Baz { }
[Foreign] class Bar implements Baz { }
\end{absexamplen}

\noindent Java:
\begin{javaexample}
package Foo;
public class Bar {

}
\end{javaexample}

\subsubsection{Discussion}
Using conventions has the disadvantage that it is not very flexible. In
particular, if classes have to be put in the same package as the generated ABS
classes this could lead to name clashes.

\PW{How to deal with sub-classing? e.g. Should we have ABS interfaces also for
abstract classes?}
      
\subsection{Use Annotations}
Use annotations on the Java level to connect Java code to ABS.
Example:
     
\noindent ABS:
\begin{absexamplen}
module Foo;
interface Baz {
  Unit print(String s);
}

[Foreign]
class Bar implements Baz { }
\end{absexamplen}     


\noindent Java:
\begin{javaexample}
package foo;

@AbsClass{"Foo.Bar"}
public class Bar {
  ABSUnit print(ABSString s2) {
    System.out.println(s2.toString());
  }
}
\end{javaexample}

\noindent Possible Java Code
\begin{javaexample}
package Test;
import abs.backend.java.afi.*;

// function definitions are put in a
// class named "Def" and must be static
@AbsDef{"Test.random"}
public static ABSInt random() {
  return new Random().nextInt();
}

public class OutputImpl {
  public ABSUnit println(ABSString s) {
    System.out.println(s.toString());
  }
}
\end{javaexample}

\end{document}
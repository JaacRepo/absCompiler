\chapter{Introduction}
This chapter describes the core ABS language as it is implemented in
the ABS tools.  The ABS language is a class-based object-oriented
language that features algebraic data types and side effect-free
functions.  Syntactically, the ABS language tries to be as close as
possible to the Java language~\cite{gosling96} so that programmers
that are used to Java can easily use the ABS language without much
learning effort.

\section{Notation}
In this chapter we often present the concrete syntax of the ABS language.\index{syntax}
To do so we use BNF\footnote{Backus-Naur Form} with the following denotations.
\begin{itemize}
\item $\OPT{x}$ denotes zero or one occurrence of $x$. \\
The same notation is used to represent an optional element in the formal system.
In effect $\OPT{x}$ corresponds to either nothing or an element of $x$.
\item $\MANY{x}$ denotes zero or more occurrences of $x$. \\
Note that in formal semantics the notation $\manymath{x}$ is used to
 represent an explicit sequence of elements $x_1,\ldots,x_n$; similarly $\manymath{x}:\manymath{T}$ represents
$x_1:T_1,\ldots,x_n:T_n$, following Pierce~\cite{Pierce:TypeSystems}.
\item $\PLUS{x}$ denotes one or more occurrences of $x$.
\item $x \bnfbar y$ means one of either $x$ or $y$.
\item \posix{$x$} denotes the POSIX character class $x$. %\todo{reference to POSIX}
%\item \verb_'x'_ denotes that \verb_x_ is a terminal symbol
%\item \bnfplus{x} denotes one or more occurrences of x. 
\item text in \TR{monospace} denotes terminal symbols.
\item text in \NT{italics} denotes non-terminals in the grammar.
%\item text in \textsf{sans serif} font denotes identifiers in an abstract grammar.
\end{itemize}

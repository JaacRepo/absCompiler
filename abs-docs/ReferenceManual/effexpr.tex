
\chapter{Expressions With Side Effects}
\label{cha:expr-with-side}

Beside pure expressions, ABS has expressions with side effects.
However, these expressions are defined in such a way that they can only have a
single side effect. This means that subexpressions of expressions can only be
pure expressions again. This restriction simplifies the reasoning about ABS
expressions.

\begin{abssyntax}
\NT{Exp}    \defn \NT{PureExp}
         ~|~ \NT{EffExp}\\
\NT{EffExp} \defn \NT{NewExp}
         \defc  \NT{SyncCall}
         \defc  \NT{AsyncCall}
         \defc  \NT{GetExp}
\end{abssyntax}

\section{New Expression}
A \emph{New Expression} creates a new object from a class name and a list of arguments. In ABS objects can be created in two different ways.
Either they are created in the current COG, using the standard \absinline{new local} expression,
or they are created in a new COG by using the \absinline{new} expression.

\begin{abssyntax}
\NT{NewExp} \defn \TR{new}\ \OPT{\TR{local}}\ \NT{TypeName}\ \TRS{(} \NT{PureExpList} \TRS{)}  
\end{abssyntax}

\begin{absexample}
new local Foo(5)
new Bar()
\end{absexample}

\subsection{Standard Object Creation}
When using the \absinline{new local} expression, the new object is created in the \emph{current} COG, i.e., the COG of the current receiver object.
Figure~\ref{fig:newExpr} illustrates this by showing two different runtime states, one before the creation of an object \absinline{b} and one after its creation.

\begin{figure}
\centering
\begin{tikzpicture}
\node[object] (a) {this:A};
\begin{pgfonlayer}{cogs}
  \node[cog, fit=(a)] (coga) {};
\end{pgfonlayer}
\node[myarrow, right=2cm] (arrow) at (coga) {\absinline{new local B()}};

\node[object, right=1cm] (a2) at (arrow.east) {this:A};
\node[object, right=1cm] (b2) at (a2) {b:B};
\begin{pgfonlayer}{cogs}
  \node[cog, fit=(a2) (b2)] (cogb) {};
\end{pgfonlayer}

\node[fit=(coga) (arrow) (cogb)] (pic) {};
\matrix[below=0.5cm] (legend) at (pic.south) {
   \node[object,legendsymbol] {~}; & \node[legendtext] {objects}; 
 & \node[cog,legendsymbol] {~}; & \node[legendtext] {COGs}; 
\\
};
\draw[gray!50] (legend.north west) -- (legend.north east);
\end{tikzpicture}
\caption{Process of creating an object inside the current COG by using the standard \absinline{new local} expression.}
\label{fig:newExpr}
\end{figure}

\subsection{COG Object Creation}
The concurrency model of ABS is based on the notion of
COGs~\cite{johnsen10fmco}.
An ABS system at runtime is a set of concurrently running COGs. A COGs can be
seen as an isolated subsystem, which has its own state (an object-heap) and its
own internal behavior. COGs are created implicitly when creating a new object by
using the \absinline{new} expression. Figure~\ref{fig:newCogExpr}
illustrates this by showing two different runtime states, one before the
creation of an object \absinline{b} using the \absinline{new} expression and
one after its creation. In the second runtime state, two COGs exists.

\begin{figure}
\centering
\begin{tikzpicture}
\node[object] (a) {this:A};
\begin{pgfonlayer}{cogs}
  \node[cog, fit=(a)] (coga) {};
\end{pgfonlayer}
\node[myarrow, right=2cm] (arrow) at (coga) {\absinline{new B()}};

\node[object, right=1cm] (a2) at (arrow.east) {this:A};
\node[object, right=1cm] (b2) at (a2.east) {b:B};

\begin{pgfonlayer}{cogs}
  \node[cog, fit=(a2)] (coga2) {};
  \node[cog, fit=(b2)] (cogb2) {};
\end{pgfonlayer}



\node[fit=(coga) (arrow) (cogb2)] (pic) {};
\matrix[below=0.5cm] (legend) at (pic.south) {
   \node[object,legendsymbol] {~}; & \node[legendtext] {objects}; 
 & \node[cog,legendsymbol] {~}; & \node[legendtext] {COGs}; 
\\
};
\draw[gray!50] (legend.north west) -- (legend.north east);
\end{tikzpicture}
\caption{Process of creating an object in a new COG by using the \absinline{new} expression.}
\label{fig:newCogExpr}
\end{figure}


\section{Synchronous Call Expression}
A \emph{Synchronous Call} consists of a target expression, a method name, and a list of argument expressions.

\begin{abssyntax}
\NT{SyncCall} \defn \NT{PureExp}\ \TRS{.}\ \NT{Identifier}\ \TRS{(} \NT{PureExpList} \TRS{)}  
\end{abssyntax}

\begin{absexample}
Bool b = x.m(5);  
\end{absexample}

\section{Asynchronous Call Expression}
An \emph{Asynchronous Call} consists of a target expression, a method name, and a list of argument expressions.
Instead of directly invoking the method, an asynchronous method call creates a new \emph{task} in the target COG, which is executed asynchronously. This means that the calling task proceeds independently after the call, without waiting for the result~\cite{johnsen10fmco}.
The result of an asynchronous method call is a future (\absinline{Fut<V>}), which can be used by the calling
task to later obtain the result of the method call.
That future is \emph{resolved} by the task that has been created in the target COG to execute the method.

\begin{abssyntax}
\NT{AsyncCall} \defn \NT{PureExp}\ \TRS{!}\ \NT{Identifier}\ \TRS{(} \NT{PureExpList} \TRS{)}  
\end{abssyntax}

\begin{absexample}
Fut<Bool> f = x!m(5);  
\end{absexample}

\section{Get Expression}\label{sec:getexpr}
A \emph{Get Expression} is used to obtain the value from a future.
The current task is blocked until the value of the future is available, i.e., until
the future has been resolved. No other task in the COG can be activated in
the meantime~\cite{johnsen10fmco}.

\begin{abssyntax}
\NT{GetExp} \defn \NT{PureExp}\ \TRS{.}\ \TR{get}
\end{abssyntax}

\begin{absexample}
Bool b = f.get;
\end{absexample}

\section{Await Expression}

A common pattern for asynchronous calls is:
\begin{enumerate}
\item Execute an asynchronous call expression, store the future in a variable
\item \TRS{await} on the future
\item Assign the result to a variable
\end{enumerate}

\begin{absexample}
Fut<A> fx = o!m();
await fx?;
A x = fx.get;
\end{absexample}

The await expression is a shorthand for this pattern.

\begin{abssyntax}
\NT{AwaitExp} \defn \TRS{await} \NT{AsyncCall}
\end{abssyntax}

The preceding example can be written as follows, without the need to
introduce a name for the future:
\begin{absexample}
A x = await o!m();
\end{absexample}

% Local Variables:
% TeX-master: "absrefmanual"
% End:
